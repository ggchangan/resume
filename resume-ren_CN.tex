% !TEX TS-program = xelatex
% !TEX encoding = UTF-8 Unicode
% !Mode:: "TeX:UTF-8"

\documentclass{resume}
\usepackage{zh_CN-Adobefonts_external} % Simplified Chinese Support using external fonts (./fonts/zh_CN-Adobe/)
% \usepackage{NotoSansSC_external}
% \usepackage{NotoSerifCJKsc_external}
% \usepackage{zh_CN-Adobefonts_internal} % Simplified Chinese Support using system fonts
\usepackage{linespacing_fix} % disable extra space before next section
\usepackage{cite}

\begin{document}
\pagenumbering{gobble} % suppress displaying page number

\name{张忍}

\basicInfo{
  \email{xueluowuhen.2007@163.com} \textperiodcentered\ 
  \phone{(+86) 185-0097-2851} \textperiodcentered\ }
 
\section{\faGraduationCap\  教育背景}
\datedsubsection{\textbf{北京航空航天大学}, 北京}{2011.9 -- 2015.3}
\textit{工学硕士}\ 计算机科学与技术
\datedsubsection{\textbf{东北师范大学}, 吉林}{2006.09 -- 2010.07}
\textit{工学学士}\ 软件工程

\section{\faCogs\ 专业技能}
% increase linespacing [parsep=0.5ex]
\begin{itemize}[parsep=0.5ex]
  \item 熟悉Golang、Java、C++,了解Python、Shell、Scala、Matlab、MFC、STL等语言
  \item 熟悉基本数据结构和算法、设计模式、多线程编程、分布式系统理念、系统设计理念
  \item 熟悉后端设计方法、Gin框架、Golang常用基础库,Golang测试框架,多次Golang后端系统实现经验
  \item 熟悉elasticsearch、clickhouse,了解lucene、kafka、redis、spark、hadoop生态
  \item 熟悉Docker、Kubenetes、Terraform、Salt,了解Prometheus,Jenkins
\end{itemize}

\section{\faUsers\ 实习/项目经历}
\datedsubsection{\textbf{FreeWheel} 北京}{2017年11月 -- 2023年5月}
\role{资深软件开发工程师}{}
ClickHouse数据分析平台 (2020年6月 -- 2023年5月)
\begin{itemize}
  \item 项目简介:ClickHouse数据分析平台,负责承载来自不同应用方的Batch数据和实时数据;为Forecast、Identity、Insight等多个团队提供实时数据查询服务。
  \item 主要工作:Batch数据分布式导入工具的开发、数据接入时间排期、性能测试工具开发、SQL性能优化;ClickHouse功能点调研、分享;表结构设计、性能测试平台开发、监控系统扩展、日常维护等
  新数据源接入:表结构设计、数据性能测试、schema升级、SQL性能优化(添加索引、视图);ClickHouse功能点调研、分享;表结构设计、性能测试平台开发、监控系统扩展、日常维护
  \item 工具和关键技术: ClickHouse、ClickHouse-exporter、Airflow、Golang、K8s、Docker、Helm、Jenkins、EKS、S3、ORC
  \item 成果:抽象出golang后端架构,可以在两人天定制web后端服务;为Forecast、Identity、Insight等多个团队提供分布式数据导入和刷新服务;系统具有良好扩展性,新的数据源的加入,只需要商量好数据源接口,将样例数据接入服务的周期从至少5个工作日降低为2个工作日;DataLoaderManager进行任务划分、资源管理、并发控制,降低了百分之50的AWS EC2费用,并且提高了数据导入的稳定性,将失败任务的重试粒度从batch级别变为表级别。
  推广ClickHouse集群一般原理和新功能分享,提高组内成员对ClickHouse功能的理解,同时优化ClickHouse平台提供数据的能力;推广ClickHouse Explain、ClickHouse Projection、二级索引等多个feature在多个team中的使用;提供ClickHouse SQL性能监控和SQL优化方法分享,将多个数据产品中高频次、运行慢的SQL从分钟级别优化到秒级(5m-10s),优化数据产品的响应能力,降低ClickHouse的负载;
  能测试平台,用于自动化ClickHouse版本升级或者新数据接入时,SQL性能测试,将此例行工作从5人天降低到2人天,并提供可视化性能对比和分析。
  负责DataLoaderManager的CI/CD、数据校验系统以及监控系统(修改了ClickHouse-explore源码、提供SQL性能监控指标以及slow sql看板等)的优化和开发。
\end{itemize}
Analytics 数据产品 (2017年11月 -- 2020年5月)
\begin{itemize}
  \item 项目简介:Analytics数据产品为客户提供数据报表服务,包括支持周期性报表导出的的Analytics和DataFeed产品;客户可以实时查看的Analytics、SupperAnalytics、DataFeed产品;CampainInsight和marketInsight等数据分析产品;以及相关监控、数据校验等;
  \item 主要工作:Analytics、DataFeed、SuperAnalytics等项目多个模块的需求评审、接口定义;Analytics、CampaignInsight、MarketInsight多个核心模块重构;Doraemon、DataFeed、SuperAnalytics的从0到1构建和代码review;推广基于gin的web后端框架;
  推广基于Salt/Terraform的CI/CD部署在DataApp团队的使用,又参与Doraemon、Analytics等多个项目基于EKS的部署;推广Airflow在DataApp团队的使用;优化Analytics基于Domain表的Looker数据建模以及相应的SQL性能。
  \item 工具和关键技术: Golang, Looker, Presto, Terraform, Salt, K8s, AWS, Gin, MySQL, Python, Airflow, Reactor, PQM
  \item 成果:产出DataFeed、CampaignInsight、SuperAnalytics、Doraemon等产品;其中DataFeed弥补了Analytics小时级别粒度数据的不足;删除时区限制产生新产品(SupperAnalytics),将Analytics能力从单个网络扩展到全网,特别提高了SRE和Service的数据分析能力;
  CampaignInsight作为过度产品成功帮公司稳固了迪士尼的客户;
  Analytics核心功能开发多格式导出(csv-->xlsx),数据格式化(日期、数字根据不同时区格式化),为公司争取了新的欧洲大客户;新的监控系统(Doraemon),将若干ESC处理能力从Engineer转到Service,此类ESC的相应时间从>1day到即时;
  Gin框架的推广以及规范的制定和推广,提高了golang后端的开发速度、优化了任务排查速度;新的CI/CD(Terraform/Salt->K8s),将Analytics的升级过程从30m降低到5s以内;SQL性能优化(整体提前2小时、Presto集群使用率下降20\%、每天节省\$200);抽象出golang后端结构,可以在两人天定制web后端服务;
\end{itemize}

\datedsubsection{\textbf{北京奇虎测腾科技有限公司} 北京}{2015年4月 -- 2017年11月}
\role{大数据平台开发工程师}{}
\begin{onehalfspacing}
数据清洗平台
\begin{itemize}
  \item 项目简介:数据清洗平台包括数据清洗和数据融合两大部分功能。数据清洗:将关系数据库中数 据导入到数据清洗平台之后,通过对数据特征分析之后,可使用规则库中规则对数据进行多次迭代式清洗,最终清洗出用户满意的数据。数据融合:将具有相同关键词的数据融合在一起。使用kafka做临时存储,elasticsearch做数据搜索和分析,hdfs做最终存储
  \item 主要工作:整体项目协调,项目整体架构,源码模块结构,数据搜索和分析模块、任务调度模块
  \item 工具和关键技术:elasticsearch、quartz、kafka、springboot、hdfs、spark
  \item 成果:完成清洗数据平台,客户比较满意
\end{itemize}
\end{onehalfspacing}
\begin{onehalfspacing}
天合大数据分析平台
\begin{itemize}
  \item 项目简介:用户将不同类型的数据——结构化数据和非结构化数据导入到天合大数据分析平台,平台通过处理分析,除了可向用户提供基本的搜索、关联图分析、地理信息数据分析等,还向用户提供定制化的功能服务,例如建立在天合之上的子项目——“数据魔方”,提供了数据录入、自定义搜索、数据浏览、关联分析、地图分析、任务管理等功能。
  \item 主要工作:设计不同类型数据的索引结构,提供索引和搜索接入,负责整个搜索模块的编码与优化;非结构化数据(包括doc,pdf,email)的导入和分析;文件服务器;地理信息数据搜索与统计;整个项目的maven化;监控系统的设计与实现;
  \item elasticsearch、netty、tika、spark、redis、kafka
  \item 成果:完成各种数据源的统一分析和可视化展示
\end{itemize}
\end{onehalfspacing}

\datedsubsection{\textbf{北京百度网讯科技技术有限公司} 北京}{2014年1月 -- 2014年6月}
\role{实习生}{}
\begin{onehalfspacing}
北京百度网讯科技技术有限公司知识搜索部(数据处理)
\begin{itemize}
  \item 概况:工作中接触的数据处理平台主要分为三类:其它部门已有大数据平台,百科部门构建的部 分大数据平台,关系数据库系统。利用现有数据平台完成向其它数据需求者提供有用数据的任 务,并接下来构建百度百科大数据系统
  \item 主要工作:通过开发实习项目“相册管理系统”了解百度LAMP开发架构和开发环境;利用其它部 门大数据平台构建数据分析任务;测试本部门的大数据平台,并从中获取特征数据;利用脚本编 程的方式为PM提供结构化符合需求数据(性能);编写脚本周期性任务。
  \item 工具和关键技术:PHP、MySQL、Hive、Hadoop、Shell
  \item 成果:实习期间为PM分析用户行为提供数据支持,对初期构建的大数据系统进行测试
\end{itemize}
\end{onehalfspacing}

\section{\faHeartO\ 获奖情况}
\datedline{\textit{北航硕士研究生学校二等奖学金两次}}{2011 -- 2012}
\datedline{东北师范大学一等奖学金两次,二等奖学金两次,优秀学生一次,曾获得国家励志奖学金、吉林省ACM IC/PC竞赛三等奖}{2006-2010}

\section{\faInfo\ 研究成果}
% increase linespacing [parsep=0.5ex]
\begin{itemize}[parsep=0.5ex]
  \item 张忍,荆凯.技术发明专利,实现非结构化文档可搜索化的一种架构,专利号(LZ1605815CN01)
  \item 荆凯,张忍.技术发明专利,不确定语言环境条件下数据规模可自动扩展地关系数据库数据可搜索化实现,专利号(LZ1605814CN01)
  \item 张忍,李帅,王莉莉,郝爱民,潘俊君.国家发明专利,人体心脏的实时逼真绘制方法,专利号 (201410768217.7)
  \item ZHANG Ren, YI Zhike, HAO Aimin, ZHEN Chenglizhao. A Real-Time Realistic Rendering Method for Active Human Heart. ICISCE 2015
\end{itemize}

%% Reference
%\newpage
%\bibliographystyle{IEEETran}
%\bibliography{mycite}
\end{document}
